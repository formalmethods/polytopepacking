\section{Encoding polytope placement into SMT}

Let's now focus on the task of placing polytopes into a parallelepiped. 
Informally, the placement is carried out by means of a vector 
for each polytope, which translates the polytope with respect to the
origin. The polytope packing problem is then the problem of finding
such placement vectors, which may or may not exist, depending on the
particular instance to solve.

Formally, given a set of polytopes $P_1, \ldots, P_n$, we need to find
placement vectors $\point{v_1}, \ldots, \point{v_n}$ such that for $0\leq i<j \leq n$
$$(\point{v_i}+P_i) \cap (\point{v_j}+P_j) \not= \emptyset$$ 
which means that for $0 \leq i<j \leq n$
$$(\point{v_j} - \point{v_i}) \not\in (P_i \ominus P_j).$$

The latter condition can be translated into a constraint satisfaction
problem. In particular $P_i \ominus P_j$ is a polytope that can be
defined by the intersection of a set of linear inequalities
\begin{eqnarray}
\nonumber
&        & c_{1_x} x + c_{1_y} y + c_{1_z} z \leq c_1 \\
\nonumber
& \wedge & c_{2_x} x + c_{2_y} y + c_{2_z} z \leq c_2 \\
\nonumber
& \wedge & \ldots \\
\nonumber
& \wedge & c_{n_x} x + c_{n_y} y + c_{n_z} z \leq c_n
\end{eqnarray}
each representing a faced of the polytope. The
area ``outside'' $P_i \ominus P_j$ is therefore
\begin{eqnarray}
\nonumber
&      & c_{1_x} x + c_{1_y} y + c_{1_z} z \geq c_1 \\
\nonumber
& \vee & c_{2_x} x + c_{2_y} y + c_{2_z} z \geq c_2 \\
\nonumber
& \vee & \ldots \\
\nonumber
& \vee & c_{n_x} x + c_{n_y} y + c_{n_z} z \geq c_n
\end{eqnarray}
Technically we should use strict inequalities $>$, however
we may allow polytopes to ``touch'' on a common point. At last
we can specify that $\point{v_i} - \point{v_j}$ is in the
area outside $P_i \ominus P_j$ with the following substitution
\begin{eqnarray}
\nonumber
&      & c_{1_x} (v_{i_x} - v_{j_x}) + c_{1_y} (v_{i_y} - v_{j_y}) + c_{1_z} (v_{i_z}-v_{j_z}) \geq c_1 \\
\nonumber
& \vee & c_{2_x} (v_{i_x} - v_{j_x}) + c_{2_y} (v_{i_y} - v_{j_y}) + c_{2_z} (v_{i_z}-v_{j_z}) \geq c_2 \\
\label{eq:smt1}
& \vee &\ldots \\                                                                                    
\nonumber
& \vee & c_{n_x} (v_{i_x} - v_{j_x}) + c_{n_y} (v_{i_y} - v_{j_y}) + c_{n_z} (v_{i_z}-v_{j_z}) \geq c_n
\end{eqnarray}
In addition to the constraints above, we need to specify the ``borders'' of the parallelepiped. Suppose
that the parallelepiped measures $l$,$w$,$h$ of length, witdth, and height respectively, and let
$x_\downarrow(P_i), x_\uparrow(P_i)$, the lowest and highest $x$ coordinate of $P_i$, 
$y_\downarrow(P_i), y_\uparrow(P_i)$, the lowest and highest $y$ coordinate of $P_i$, 
$z_\downarrow(P_i), z_\uparrow(P_i)$, the lowest and highest $z$ coordinate of $P_i$. Then we
need to encode for all $i$
\begin{eqnarray}
\nonumber
&& 0 \leq v_{i_x} + x_\downarrow(P_i) \wedge v_{i_x} + x_\uparrow(P_i) \leq l \\
\label{eq:smt2}
&& 0 \leq v_{i_y} + y_\downarrow(P_i) \wedge v_{i_y} + y_\uparrow(P_i) \leq w \\
\nonumber
&& 0 \leq v_{i_z} + z_\downarrow(P_i) \wedge v_{i_z} + z_\uparrow(P_i) \leq h
\end{eqnarray}
By encoding (\ref{eq:smt1}) and (\ref{eq:smt2}) into the SMT2~\cite{SMTLIB} language, we can find
values of $\point{v_i}$ for each $P_i$ that represent a polytope placement such
that the polytopes (may touch but) do not intersect and such that
is contained in the given parallelepiped.
